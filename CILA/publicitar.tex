\documentclass[10pt,a4paper]{article}
\usepackage[latin1] {inputenc}

\begin{document}

\pagestyle{empty}

\begin{center}
\textbf{Publicitando el CILA}
\end{center}


\section{El proyecto}

El  {\bf  C}urso  de  {\bf  I}ntroducci�n de  {\bf  L}inux  para  {\bf
A}lumnos, tambi�n conocido  como {\bf CILA}, es  un ambicioso proyecto
de documentaci�n en  el que se pretende dar una  introducci�n al mundo
del Linux tratando de abarcar el mayor n�mero de temas sin llegar a la
profundidad  en cada  uno de  ellos, porque  entonces podr�amos  estar
hablando, no de uno, sino de muchos libros.


\section{Motivos a ponerlo en marcha}

La idea surgi� de Miguel �ngel Vilela, miembro del {\bf GULiC} -- {\bf
G}rupo de {\bf U}suarios de {\bf L}inux de {\bf C}anarias -- a ra�z de
querer impartir  un curso  introductorio de Linux  para alumnos  de la
Facultad de Matem�ticas de la Universidad  de La Laguna con el fin que
pudieran realizar sus pr�cticas de curso en Linux con sus herramientas
disponibles.

\subsection*{Tiempo empleado}
\section{Objetivos logrados}
\subsection*{Logros importantes}
\section{Dificultades aparecidas}
\subsection*{Cosas por hacer}
\section{�A qui�n beneficia el proyecto?}
\subsection*{�Qui�nes pueden tener inter�s en los resultados?}
\section{�Qui�nes pueden participar en el proyecto?}
\end{document}
