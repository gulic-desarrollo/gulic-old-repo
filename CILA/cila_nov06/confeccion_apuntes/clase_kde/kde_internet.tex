\section{KDE Internet}
\subsection{Navegador Firefox}
\frame
{
	\begin{center}
    \includegraphics[height=6cm]{./imgs/konqueror-gulic}
  \end{center}
}
\subsection{Chat y Clientes de Mensajería Instántanea}
\frame
{
	\frametitle{konversation, cliente de IRC}
	\begin{center}
		\includegraphics[height=6cm]{./imgs/konversation}
	\end{center}
}
\frame
{
	\frametitle{Kopete, cliente de mensajería instantánea}
	\begin{center}
		\includegraphics[height=6cm]{./imgs/kopete}
	\end{center}
}
\subsection{Cliente de correo}
\frame
{
	\frametitle{Kmail}
	\begin{center}
		\includegraphics[height=6cm]{./imgs/kmail}
	\end{center}
}
\subsection{Gestor de Información Personal}
\frame
{
	\frametitle{Kontact}
	\begin{center}
		\includegraphics[height=6cm]{./imgs/kontact2}
	\end{center}
}
\subsection{Agregador de noticias}
\frame
{
	\frametitle{Akregator}
	Akregator es un Agregador de noticias que soporta feeds (suministro de datos) en formato RSS y Atom.
	\begin{center}
		\includegraphics[height=4cm]{./imgs/akregator}
	\end{center}
}

\subsection{Internet no KDE}
\frame{
	\frametitle{Navegador Mozilla Firefox}
	\begin{center}
		\includegraphics[height=4cm]{./imgs/firefox}
	\end{center}
	En https://addons.mozilla.org/:
	\begin{itemize}
		\item Themes: Distindos vestidos para el navegador firefox.
		\item Plugins: Acrobat Reader, Flash Player, Adblock...
		\item Extensiones: Fasterfox, ViewSourceWith, WeatherBug, Mouse Gestures...
	\end{itemize}
}
\frame
{
	\frametitle{Amsn}
	Cliente mensajería instántanea muy similar al msn para conectar con el protocolo Msn Messenger.
	\begin{center}
		\includegraphics[height=4cm]{./imgs/amsn}
	\end{center}
}
\frame
{
	\frametitle{Amule}
	 aMule es un programa de intercambio P2P libre y multiplataforma, similar al conocido eMule que funciona tanto con la red eDonkey como con Kademlia.
	\begin{center}
		\includegraphics[height=4cm]{./imgs/amule}
	\end{center}

}
