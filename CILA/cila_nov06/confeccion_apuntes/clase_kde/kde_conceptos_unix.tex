\section{KDE Conceptos Básicos Unix}
\subsection{Usuarios y grupos}
\frame
{
	\frametitle{Usuarios y grupos}
	\begin{center}
		\includegraphics[width=8cm]{./imgs/usuarios-grupos}
	\end{center}
}
\subsection{Permisos en Linux}
\frame
{
	\frametitle{Permisos en linux}
	\begin{center}
		\begin{itemize}
			\item<1->{Se diferencian entre prvilegios para directorios y privilegios para ficheros}
				\begin{itemize}
					\item<2->{En ficheros: lectura, escritura y ejecución}
					\item<3->{En directorios: ver el contenidos, modificar el contenido, entrar en el directorio}
					\item<4->{Permisos especiales: +s, +t ...}
				\end{itemize}
			\item<5->{Se pueden especificar para el dueño del fichero, el grupo al que pertenece el fichero y el resto de usuarios}
		\end{itemize}
	\end{center}
}
\frame
{
	\frametitle{Modificar los privilegios de los ficheros}
	\begin{center}
		\includegraphics[height=3.5cm]{./imgs/konqueror-preferencias}
		\includegraphics[height=3.5cm]{./imgs/konqueror-privilegios}
	\end{center}
}

\subsection{El árbol de directorios}
\frame
{
	\frametitle{El árbol de directorios}
	\begin{center}
		\includegraphics[height=6cm]{./imgs/konqueror-arbol}
	\end{center}
}
