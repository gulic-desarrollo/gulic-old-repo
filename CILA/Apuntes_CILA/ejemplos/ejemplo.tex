% BEGIN: Ejemplo 1 de LaTeX para CILA

% Define el  tipo de documento  como "article" (art�culo)  y especifica
% las opciones "a4paper" (tama�o de  papel A4), "12pt" (tama�o de letra
% a 12 puntos) y "twoside" (para imprimir a doble cara)
\documentclass[a4paper,12pt,twoside]{article}

% Utiliza  el  paquete "inputenc"  con  la  opci�n "latin1",  esto  nos
% permite teclear  las "�" y las  tildes sin tener que  preocuparnos de
% nada, porque el charset que estamos  usando el latin1 y se lo decimos
% a LaTeX
\usepackage[latin1]{inputenc}

% Utiliza el  paquete "babel" con  la opci�n "spanish", lo  que incluye
% entre  otras  cosas  patrones  de silabeo,  traducci�n  de  elementos
% insertados  autom�ticamente,  como  "�ndice   general"  en  lugar  de
% "Contents"
\usepackage[spanish]{babel}

% Utiliza el  paquete amssymb,  American Mathematical  Society SYMbols,
% (s�mbolos de  la Sociedad Matem�tica  Americana). Es este  paquete se
% encuentra  por ejemplo  el s�mbolo  de "isomorfo  a" utilizado  en el
% Teorema de Isomorf�a.
\usepackage{amssymb}

% Utiliza el paquete  "eurosym" para proporcionar el  s�mbolo del euro.
% Con este  paquete podemos escribir el  s�mbolo del euro con  la orden
% \euro
\usepackage{eurosym}

% Ordenamos a LaTeX que no numere  las p�ginas, ya que tenemos una s�la
% p�gina y un "1" solitario no queda muy bien.
\pagestyle{empty}

% Evita que  LaTeX introduzca  espacios mayores de  lo normal  tras los
% finales de las oraciones. Una pijada.
\frenchspacing

\begin{document} % Comienza el documento

% Esta frase estar� centrada y con un tama�o de letra mayor
\begin{center}
{\large Curso de Introducci�n a Linux para Alumnos}
\end{center}

Esto es un  peque�o ejemplo de {\LaTeX}, el m�s  potente procesador de
textos. La mayor�a  de apuntes y ex�menes de Matem�ticas  que vemos en
la  Facultad est�n  escritos  en {\LaTeX}.  Por  cierto, con  {\LaTeX}
tambi�n estamos preparados para la llegada del \euro

A continuaci�n algunos ejemplos de f�rmulas matem�ticas:

% Comenzamos una descripci�n de varios "items"
\begin{description}

\item [Definici�n de l�mite (An�lisis Matem�tico I)]
$$
\lim_{x \longrightarrow a} = l \iff 
\forall \, \varepsilon > 0 \,
\, \exists \, \delta > 0 \, / \,
\, 0 < \| x - a \| < \delta \,
\Longrightarrow \, \| f(x) - l \| < \varepsilon
$$

\item [Teorema Generalizado de Cauchy (An�lisis Matem�tico I)]
Si $f$ y $g$ tienen derivadas cont�nuas hasta el orden $n(n-1)$ en 
el intervalo $\lbrack a , b \rbrack$ y adem�s $\forall \, x \in (a,b)
\, \, \exists \, f^{n)}(x), g^{n)}(x)$, entonces
$\exists \, c \in (a,b) \, /$
$$
\bigg( f(b) - \sum_{k=0}^{n-1} \frac{f^{k)}(a)}{k!}(b-a)^k \bigg) g^{n)}(x) =
f^{n)}(x) \bigg( g(b) - \sum_{k=0}^{n-1} \frac{g^{k)}(a)}{k!}(b-a)^k \bigg) 
$$

\item [$1^{er}$ Teorema de Isomorf�a (�lgebra I)]
Sean $G$, $G'$ grupos, $f : G \longrightarrow G'$
homomorfismo de grupos. Entonces
$$
\frac{G}{Ker(G)} \, \thickapprox \, Im f
$$

\item [Funciones Eulerianas: Gamma y Beta (An�lisis Matem�tico II)]
$$
\Gamma(p) = \int_{0}^{+\infty} e^{-x} x^{p-1} dx \quad \forall \, p > 0
$$
$$
\beta(p,q) = \int_{0}^{1} x^{p-1} (1-x)^{q-1} dx
$$

\item [Y por �ltimo, un ejemplo denso]
$$
\sum \limits_{n = 0}^{\infty} 
\left(
  \frac
    {\int \limits_{-\infty}^{+\infty}
      {\left\lceil 
        \frac
          {\sin 
      	    \left[ 
    	      8 \frac{\pi}{3}^3
    	    \right]}
          {\arctan \left(
    	    \sqrt[3]{ 2 \cdot \sin {(x)} }
    	  \right)}
      \right\rceil 
      dx}
    }
    {\lim \limits_{x \to n^2}
      \left(
        \vert{
          \frac
    	    {\log
    	      {\frac
    	        {\pi}
      	        {x^4}
    	      }
    	    }
            {e^{
    	      \frac
    	        {n + 1}
    	        {n - 1}
    	      }
            }
        \vert}
      \right)
    }
\right)
=
\left|
  \begin{array}{cccc}
    m_{(i,j)}   & m_{(i,j+1)}       & \ldots   & m_{(i,n)}     \\
    m_{(i+1,j)} & m_{(i+1,j+1)}     & \ldots   & m_{(i+1,n)}   \\
    \vdots        & \vdots              & \ddots & \vdots          \\
    m_{(n,j)}   & m_{(i + n,j + 1)} & \ldots   & m_{(n,n)}
  \end{array}
\right|
$$

\end{description}

\end{document} % Termina el documento

% END: Ejemplo 1 de LaTeX para CILA
