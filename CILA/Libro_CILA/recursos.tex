%Autor: todos
  
\chapter{Recursos en internet}
  

La mayor�a  de los recursos que  podamos necesitar para usar  Linux se
encuentran  distribuidos por  Internet,  y una  enorme cantidad  est�n
disponibles en espa�ol.

\begin{itemize}

\item \htmlurl{http://www.debian.org} -- El Proyecto Debian es una asociaci�n
  de  personas  que  han  hecho  causa com�n  para  crear  un  sistema
  operativo (SO)  libre. Este  sistema operativo  que hemos  creado se
  llama Debian GNU/Linux, o simplemente Debian para acortar.

\item \htmlurl{http://www.redhat.es} --  La  primera  distribuci�n Linux  en
  reorientarse hacia los usuarios finales.

\item \htmlurl{http://www.linux-mandrake.com/es/} -- Linux-Mandrake es
  un amigable Sistema Ope\-ra\-ti\-vo Linux. Es  muy f�cil de usar, tanto en
  el  hogar u oficina  como en servidores. Est�  disponible  en  forma
  gratuita en varios idiomas alrededor del mundo.

\item \htmlurl{http://www.ututo.org} -- Un GNU/Linux Simple.

\item \htmlurl{http://www.linux-es.com} -- Existen muchos lugares en
  Internet  dedicados  a  LINUX,  pero  la mayor�a de ellos est�n en
  ingl�s. Estas p�ginas  pretenden  ser un  punto  de  partida  para
  aquellos  que necesitan encontrar informaci�n sobre este sistema y
  en  la  medida de lo posible se ha intentado que la mayor�a de los
  enlaces y contenidos sean en castellano.

\item \htmlurl{http://www.gnu.org/home.es.html} -- El  Projecto  GNU
  comenz� en  1984 para  desarrollar un sistema  operativo tipo Unix
  completo, el cual  es  software libre:  El  sistema  GNU. Variantes
  del  sistema GNU, utilizando Linux  como  kernel,  son  ampliamente
  usadas, y aunque frecuentemente llamadas ``Linux'', dichas variantes
  deber�an referirse m�s exactamente como sistemas GNU/Linux.

\item \htmlurl{http://lucas.hispalinux.es}  --  Proyecto LuCAS  -  La
  mayor biblioteca en espa�ol dedicada a GNU/LiNUX de todo el planeta

\item \htmlurl{http://www.insflug.org}  --  En  el INSFLUG  se  coordina
  la traducci�n ``oficial'' de documentos  breves, como los COMOs  y
  PUFs o  Preguntas de Uso Frecuente, las FAQs  en ingl�s. Esperamos
  que la informaci�n que encuentre aqu� le sea de utilidad.

\item \htmlurl{http://www.gnu.org/software/emacs/emacs.html} --  Es un
  editor de  pantalla y  ambiente  para c�mputo  de  tiempo real,
  extensible  y  personalizable.  Ofrece Lisp (finalmente integrado al
  editor)  para escribir  extensiones  y proporciona  una interfaz  al
  sistema de ventanas X.

\item \htmlurl{http://www.vim.org} -- VIM es  una versi�n mejorada del
  editor VI, uno de los editores de texto est�ndar en los sistemas UNIX.
  VIM a�ade muchas de las  caracter�sticas que se  esperan en  un editor:
  Deshacer ilimitado, coloreado de sintaxis, GUI, y mucho m�s.

\item \htmlurl{http://www.escomposlinux.org/sromero/linux/vim-tut.html}  --
  Usar VIM  (el editor),  de la  zarpa de Santiago  Romero. Vim  es el
  t�pico  editor que  todos queremos  usar cuando  seamos mayores.  En
  particular parece  bastante complicado  al principio pero  cuando lo
  usas no puedes vivir sin �l :)

\item \htmlurl{http://www.escomposlinux.org/sromero/linux/mutt-tut.html}  --
  Usar MUTT (MailReader), del mismo autor que el anterior.

\item \label{putty}\htmlurl{http://www.chiark.greenend.org.uk/~sgtatham/putty/} --
  Implementaci�n libre de un cliente TELNET/SSH para sistemas operativos
  Microsoft� Windows�. Escrito y mantenido por Simon Tatham.

\item \htmlurl{http://www.escomposlinux.org/sromero/linux/} -- S.O.S. Linux.

\item \htmlurl{http://www.geocities.com/Athens/Temple/2269/} --  Tutorial de
  C/C++

\item \htmlurl{http://www.fie.us.es/docencia/publi/JAVA/} -- Tutorial de Java

\item \htmlurl{http://apolo.us.es/CervanTeX/} -- Informaci�n LaTeX en espa�ol

\item \htmlurl{ftp://ftp.cma.ulpgc.es/pub/software/TeX/tex/latex2e/doc/ldesc2e/mix/ldesc2e.pdf} -- Una descripci�n de LaTeX, por Tom�s Bautista y cia.

\item \htmlurl{http://www.lyx.org}  --  P�gina del proyecto LyX 

\item \htmlurl{http://www.octave.org}  --  P�gina del proyecto Octave

\item \htmlurl{http://www.gnuplot.org}  --  P�gina del proyecto Gnuplot

\item \htmlurl{http://www.r-project.org} -- P�gina del proyecto R

\item \htmlurl{http://www-rocq.inria.fr/scilab/}  --  P�gina del proyecto SciLab

\item \htmlurl{http://www.lysator.liu.se/\~alla/dia/} --  P�gina del proyecto DIA

\item \htmlurl{http://www.gimp.org} --  P�gina del proyecto GIMP

\end{itemize}


