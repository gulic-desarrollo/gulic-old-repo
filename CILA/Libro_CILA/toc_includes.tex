%\incluye{administracion}

% M�dulos
% I.   Entorno GNU/Linux (3 d�as, 15 horas en clases de 5 horas)
% II.  Party (2 d�as, 20 horas en jornadas de 10 horas)
% III. Documentaci�n (5 d�as, 20 horas, en clases de 4 horas)
% IV.  Matem�ticas (3 d�as, 12 horas, en clases de 4 horas)
% V.   Programaci�n (5 d�as, 20 horas, en clases de 4 horas)
% VI.  Programaci�n avanzada (?? horas)

% M�dulo I:   Entorno GNU/Linux
% 1. Historia de GNU/Linux
% 2. El entorno X-Window (Gnome, KDE)
% 3. El entorno del int�rprete de comandos (con mtools)
% 4. Usando internet (mozilla, konqueror, kmail, gftp, mutt, SSH)
% 5. Aplicaciones (gnotepad, abiword, openoffice)
% 6. Documentaci�n y sistemas de ayuda
\part{Entorno GNU/Linux}
%= \incluye{introduccion}
%= \incluye{comandos}
%= \incluye{xwindow}
%= \incluye{editores}
%= \incluye{internet}
%= \incluye{aplicaciones}
%= \incluye{documentacion}

% M�dulo II:  Party
% 1. Sistema base (particiones)
% 2. Hardware y kernel (make [menu|x]config && make-kpkg)
% 3. Software para los m�dulos.
% 4. Administraci�n b�sica (adduser, floppy, cdrom, APT)
% 5. Paquetes en fuentes (bajar, compilar, instalar Xine+dvdnav)
% 6. Software adicional (a gusto del consumidor)
\part{Instalaci�n de GNU/Linux}
%= \incluye{instalacion}
%%% --- BUGGY --- \incluye{configuracion}
%= \incluye{administracion}
%= \incluye{bash}

% M�dulo III: Edici�n y maquetaci�n de documentos
% 1. Edici�n de gr�ficos (DIA, QCad, The GIMP)
% 2. OpenOffice
% 3. HTML (lenguaje, bluefish, quanta)
% 4. LyX
% 5. LaTeX
\part{Edici�n de gr�ficos y documentos}
%%%\incluye{graficos}
\incluye{gimp}
%= \incluye{staroffice}
%\incluye{openoffice}
%= \incluye{html}
%= \incluye{lyx}
\incluye{latex}

% M�dulo IV:  Matem�ticas
% 1. GNUplot
% 2. Octave (SciLab)
% 3. R
% 4. Yacas
% 5. Maxima
\part{Herramientas matem�ticas}
%= \incluye{octave}
%= \incluye{gnuplot}
%= \incluye{r}
%= \incluye{yacas}

% M�dulo V:   Programaci�n
% 1. Editores de texto (gvim, xemacs)
% 2. Bash
% 3. GNU Fortran 77
% 4. FreePascal
% 5. GNU C/C++
% 6. Java
% 7. GNU Make
% 8. Depuradores (gdb y ddd)
%\thispagestyle{empty}
\part{Programaci�n}
%= \incluye{freepascal}
%= \incluye{gnufortran}
%= \incluye{gnuc}
%= \incluye{gnumake}
%= \incluye{depuradores}
%\incluye{autoconf}
%= \incluye{gprof}
%= \incluye{java}
%= \incluye{regex}

%
% NO SE ENTRETENGAN CON ESTO TODAV�A
%
% M�dulo VI:  Programaci�n avanzada
% 1. PHP + [My|Postgre]SQL
% 2. Perl
% 3. Python (NumPy) 
% 4. GUIs toolkits: Tk, QT, Wx, Gtk, Glade
% 5. XML
%\part{Programaci�n avanzada}
%= \incluye{php}
%\incluye{perl}
%\incluye{python}
%\incluye{toolkits}
%\incluye{xml}
%\indluye{ides}


% Ap�ndices y referencias
\appendix
%= \incluye{recursos}
%= \incluye{fdles}
