%% Autores:
%%      lcabrera
%%      li-po

%% Versi�n
%% $Id$

%% Algunas aplicaciones a detallar:
%% gnotepad,acroread, ghostview, gnumeric, gqview, videolan client, xine

\chapter{Aplicaciones diversas} 

En  este  apartado  trataremos,   de  manera  muy  superficial,  algunas
aplicaciones de uso cotidiano en Linux.

%% Primera  secci�n
\section{Aplicaciones Gr�ficas} 

Las aplicaciones gr�ficas  disponibles hoy d�a cubren  un amplio abanico
de actividades. Aqu� expondremos solo unas pocas de ellas.

\subsection{Internet}
Dentro  de  las  diversas  aplicaciones relacionadas  con  Internet  que
podemos  usar a  diario,  podemos resaltar  aquellas  que se  encuentran
orientadas a  la comunicaci�n directa  entre personas, como los  son los
programas de Mensajer�a Instantanea o los clientes de IRC

\subsubsection*{Gaim}

Entre   la  surtida   gama  de   clientes  de   Mensajer�a  Instantanea,
destacaremos  el  programa  Gaim.  La caracteristica  que  nos  mueve  a
destacar este programa  en particular es la gran  cantidad de protocolos
soportados  en  un  �nico  cliente.   Con  este  cliente  de  Mensajer�a
Instantanea podremos conectarnos  con los servicios de  MSN, Yahoo, AOL,
ICQ,  Jabber, Napster  o al  propio IRC,  por citar  algunos, por  citar
algunos de los protocolos soportados.

(Poner una captura del Gaim)

\subsubsection*{XChat/KVirc}
Otro de los servicios que se suelen utilizar con bastante frecuencia, se
encuentran las �consultas� al IRC. Este protocolo nos permite estar
interconectados con otros grupos de usuarios, de una manera muy din�mica.
En Linux, en el apartado de clientes de IRC gr�ficos, destacan especialmente
dos clientes: XChat y Kvirc

Xchat es un cliente de IRC muy flexible, que nos permite mantener sesiones no
solo en varios canales al mismo tiempo, sino que incluso nos permite conectar a
varios servidores de IRC al mismo tiempo, todo ello de una manera muy
intuitiva.

(Poner captura del XChat)

De la misma manera, el Kvirc es otro cliente gr�fico capaz de satisfacer al m�s
exigente de los usuarios. Entre otras cosas destaca por las ayudas que presta a
aquellos que les guta disfrutar haciendo �scripts� para los clientes de IRC

(Poner una captura del Kvirc)


\subsection{Multimedia}
\subsubsection*{Xine}
%\subsubsection*{Aviplay}

\subsection{Ofim�tica}
%\subsubsection*{Koffice}
\subsubsection*{OpenOffice}

\subsection{Gr�ficos}
\subsubsection*{Gimp}
\subsubsection*{eog}
\subsubsection*{gqview}
\subsubsection*{ghostview}

\subsection{Desarrollo}
\subsubsection*{Glade/Kdevelop}
\subsubsection*{Cervicia/gcvs}
\subsubsection*{Quanta/Bluefish}

\subsection{Editores}
\subsubsection*{AbiWord}
\subsubsection*{Kwrite}
\subsubsection*{gnotepad}
%\subsubsection*{Kwordtrans}

%\subsection{Sistema}
%\subsubsection*{gmemusage}
%\subsubsection*{gr_monitor}

\subsection{Aplicaciones en general}
\subsubsection*{Creaci�n de CD's}
\paragraph*{GCombust}
%\paragraph*{XCDRoad?}
%\paragraph*{KreateCD}
%\paragraph*{Kover}

%\subsubsection*{Por Definir}

%% Segunda secci�n
\section{Aplicaciones no Gr�ficas}

Se  puede  afirmar  que  por cada  aplicaci�n  gr�fica  disponible  en
Linux,  tenemos 50  utilidades  disponibles desde  la terminal.  Estas
aplicaciones son  el verdadero motor  de todos los  sistemas derivados
de  Unix. Con  ellas  se  pueden llevar  a  cabo  infinidad de  tareas
importantes, desde ordenar  cualquier tipo de archivo  de datos, hasta
programar  el  encendido  de  una cafetera  para  cuando  nos  estemos
quedando dormidos. Todas ellas pueden  ser lanzadas desde una terminal
en modo gr�fico, de manera que tambi�n las podemos utilizar desde este
entorno.


%\subsection{Desarrollo}
%\subsubsection*{GCC}
%\subsubsection*{make}
%\subsubsection*{gettext}

%\subsection{Editores}
%\subsubsection*{emacs}
%\subsubsection*{vim}

\subsection{Gr�ficos}
\subsubsection*{ImageMagick}
\subsubsection*{Otros Filtros}

%\subsection{Internet}
%\subsubsection*{iptraf}
%\subsubsection*{ntop}

\subsection{Multimedia}
\subsubsection*{mpg321}
\subsubsection*{aumix}
\subsubsection*{zgv}
\subsubsection*{aatv}

%\subsection{Ofim�tica}
%\subsubsection*{bc}
%\subsubsection*{}

\subsection{Sistema}
\subsubsection*{top}
\subsubsection*{crontab}
\subsubsection*{at}
\subsubsection*{uptime}

