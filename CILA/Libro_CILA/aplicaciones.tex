%% Autores:
%%      lcabrera
%%      li-po

%% Versi�n
%% $Id$

%% Algunas aplicaciones a detallar:
%% gnotepad,acroread, ghostview, gnumeric, gqview, videolan client, xine

\chapter{Aplicaciones diversas}  En este apartado trataremos,  de manera
muy superficial, algunas aplicaciones de uso cotidiano en Linux.

%% Primera  secci�n

\section{Aplicaciones Gr�ficas} Las  aplicaciones  gr�ficas  disponibles
hoy d�a cubren  un amplio abanico de actividades.  Aqu� expondremos solo
unas pocas de ellas.


\subsection{Desarrollo}
\subsubsection*{}
\subsubsection*{}

\subsection{Editores}
\subsubsection*{}
\subsubsection*{}

\subsection{Gr�ficos}
\subsubsection*{}
\subsubsection*{}

\subsection{Internet}
\subsubsection*{}
\subsubsection*{}

\subsection{Multimedia}
\subsubsection*{}
\subsubsection*{}

\subsection{Ofim�tica}
\subsubsection*{}
\subsubsection*{}

\subsection{Sistema}
\subsubsection*{}
\subsubsection*{}

\subsection{Aplicaciones en general}
\subsubsection*{Creaci�n de CD's}
\subsubsection*{}

%% Segunda secci�n


\section{Aplicaciones no Gr�ficas}
Se puede  afirmar que por  cada aplicaci�n gr�fica disponible  en Linux,
tenemos 50 utilidades disponibles  desde la terminal. Estas aplicaciones
son el  verdadero motor  de todos  los sistemas  derivados de  Unix. Con
ellas se  pueden llevar  a cabo infinidad  de tareas  importantes, desde
ordenar cualquier tipo de archivo de datos, hasta programar el encendido
de una cafetera para cuando nos estemos quedando dormidos :) Todas ellas
pueden ser  lanzadas desde una terminal  en modo gr�fico, de  manera que
tambi�n las podemos utilizar desde este entorno.


\subsection{Desarrollo}
\subsubsection*{GCC}
\subsubsection*{make}
\subsubsection*{gettext}

\subsection{Editores}
\subsubsection*{emacs}
\subsubsection*{vim}

\subsection{Gr�ficos}
\subsubsection*{ImageMagick}
\subsubsection*{Otros Filtros}

\subsection{Internet}
\subsubsection*{iptraf}
\subsubsection*{ntop}

\subsection{Multimedia}
\subsubsection*{mpg321}
\subsubsection*{aumix}
\subsubsection*{zgv}
\subsubsection*{aatv}

\subsection{Ofim�tica}
\subsubsection*{bc}
\subsubsection*{}

\subsection{Sistema}
\subsubsection*{top}
\subsubsection*{crontab}
\subsubsection*{at}
\subsubsection*{uptime}

